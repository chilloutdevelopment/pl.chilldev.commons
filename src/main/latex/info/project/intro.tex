%%
% This file is part of the ChillDev-Commons.
%
% @license http://mit-license.org/ The MIT license
% @author Rafał Wrzeszcz <rafal.wrzeszcz@wrzasq.pl>
% @copyright 2014 © by Rafał Wrzeszcz - Wrzasq.pl.
% @version 0.0.1
% @since 0.0.1
% @category ChillDev-Commons
%%

% Section heading.
\section{Introduction}

ChillDev-Commons is common routines and data structures library for Java. It is written by Rafał Wrzeszcz - Wrzasq.pl (Chillout Development is not a legitimate company in meaning of law - it's a mark of Rafał Wrzeszcz - Wrzasq.pl. For details see \url{http://wrzasq.pl/} and \url{http://chilldev.pl/}).

\subsection{Dependencies}

To work with any ChillDev-Commons module you will need to have following dependencies installed:

\begin{description}
    \item[Git] \hfill \\
    all project sources and additional resources are stored in Git SCM and this is so far the only distribution channel, so you will need Git at least to obtain sources;

    \item[Maven] \hfill \\
    is used for build process automation;

    \item[TeX Live] \hfill \\
    will be needed for processing \LaTeX\ documentation sources;

    \item[Java7 JRE and JDK] \hfill \\
    Java is essential for running ChillDev-Commons since this is the technology used for developing it - normaly you will need only JRE for running it, but in case of any source code modifications, you will need JDK to rebuild it;

    \item[JUnit 4] \hfill \\
    this is a unit testing framework used by ChillDev-Commons, you will only need it for development to run automated tests;

    \item[Cobertura] \hfill \\
    is a test code coverate tool that marks parts of code that are uncovered by tests yet, you will only need it for development.
\end{description}

\subsection{Abstract files}

In your distribution there should also exist following files with basic abstract information about the project and product:

\begin{description}
    \item[LICENSE] \hfill \\
    LGPLv3 license under which ChillDev-Commons is distributed;

    \item[README.md] \hfill \\
    brief description of project and usage (in \emph{Markdown} format).
\end{description}

Submodules can extend this list with files providing more detailed description of their purpose and usage.
